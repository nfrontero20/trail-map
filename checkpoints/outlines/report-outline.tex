% !TeX program = pdfLaTeX
\documentclass[12pt]{article}
\usepackage{amsmath}
\usepackage{graphicx,psfrag,epsf}
\usepackage{enumerate}
\usepackage{natbib}
\usepackage{textcomp}
\usepackage[hyphens]{url} % not crucial - just used below for the URL
\usepackage{hyperref}
\providecommand{\tightlist}{%
  \setlength{\itemsep}{0pt}\setlength{\parskip}{0pt}}

%\pdfminorversion=4
% NOTE: To produce blinded version, replace "0" with "1" below.
\newcommand{\blind}{0}

% DON'T change margins - should be 1 inch all around.
\addtolength{\oddsidemargin}{-.5in}%
\addtolength{\evensidemargin}{-.5in}%
\addtolength{\textwidth}{1in}%
\addtolength{\textheight}{1.3in}%
\addtolength{\topmargin}{-.8in}%

%% load any required packages here




\usepackage{booktabs}
\usepackage{longtable}
\usepackage{array}
\usepackage{multirow}
\usepackage{wrapfig}
\usepackage{float}
\usepackage{colortbl}
\usepackage{pdflscape}
\usepackage{tabu}
\usepackage{threeparttable}
\usepackage{threeparttablex}
\usepackage[normalem]{ulem}
\usepackage{makecell}
\usepackage{xcolor}

\begin{document}


\def\spacingset#1{\renewcommand{\baselinestretch}%
{#1}\small\normalsize} \spacingset{1}


%%%%%%%%%%%%%%%%%%%%%%%%%%%%%%%%%%%%%%%%%%%%%%%%%%%%%%%%%%%%%%%%%%%%%%%%%%%%%%

\if0\blind
{
  \title{\bf Capstone project - outline}

  \author{
        Nicole Frontero \\
    Department of Mathematics and Statistics, Amherst College\\
      }
  \maketitle
} \fi

\if1\blind
{
  \bigskip
  \bigskip
  \bigskip
  \begin{center}
    {\LARGE\bf Capstone project - outline}
  \end{center}
  \medskip
} \fi

\bigskip
\begin{abstract}
Once I've written the report I'll add an abstract
\end{abstract}

\noindent%
{\it Keywords:} I'll need to add keywords
\vfill

\newpage
\spacingset{1.45} % DON'T change the spacing!

\emph{Note that I will include a table of contents}

\hypertarget{introduction}{%
\section{Introduction}\label{introduction}}

Here I will discuss why I chose this topic and will preview what the
report contains.

\hypertarget{motivation}{%
\subsection{Motivation}\label{motivation}}

Motivations include:

\begin{itemize}
\tightlist
\item
  I have long had an interest in maps (since childhood)
\item
  I wish I had access to a Shiny app like this when I was at Amherst
\item
  I have no exposure to working with GIS data so this will be a
  challenge
\end{itemize}

\hypertarget{overview-of-the-paper}{%
\subsection{Overview of the paper}\label{overview-of-the-paper}}

I will go through the structure of the paper (subject to change).

\hypertarget{expository-review}{%
\section{Expository review}\label{expository-review}}

\hypertarget{basics-of-working-with-geospatial-data}{%
\subsection{Basics of working with geospatial
data}\label{basics-of-working-with-geospatial-data}}

Here I will cover the very basics of geospatial data.

\hypertarget{why-is-it-worthwhile-to-work-with-geospatial-data}{%
\subsubsection{Why is it worthwhile to work with geospatial
data?}\label{why-is-it-worthwhile-to-work-with-geospatial-data}}

MDSR section 17.1 provides a good example of why geospatial data can be
helpful. I plan to come up with my own example (similar to the cholera
deaths example) and/or list some examples of how geospatial data is
ubiquitous in modern society: for example, think Google Earth, weather
radar maps, GPS systems, etc.

\hypertarget{geospatial-data-files}{%
\subsubsection{Geospatial data files}\label{geospatial-data-files}}

I will first emphasize that there are \emph{a lot} of different GIS data
file types. I might give some as examples. Then, I will focus primariy
on the file types that I will be utilizing (shp, shx, dbf). I will
discuss the functions/packages I will heavily relying upon to read the
files/use them in R.

\hypertarget{projections}{%
\subsubsection{Projections}\label{projections}}

I will give a general overview of the concept of projections. I will
also include some images that show drastically different portrayals of
Earth in 2D depending upon the projections used.

\hypertarget{geospatial-computations}{%
\subsection{Geospatial computations}\label{geospatial-computations}}

I will try to emphasize in this section that geospatial computations
sound scary but aren't!

\hypertarget{computing-distance}{%
\subsubsection{Computing distance}\label{computing-distance}}

I'll go into how distance is computed.

\hypertarget{elevation-charts}{%
\subsubsection{Elevation charts}\label{elevation-charts}}

I'll go into how elevation is computed.

\hypertarget{shiny-app}{%
\section{Shiny app}\label{shiny-app}}

The Shiny app portion of the report will include, for each aspect of the
app:

\begin{itemize}
\tightlist
\item
  A brief introduction as to why I am including that aspect in the app
\item
  Coding that I did
\item
  A screen shot of that aspect of the app
\item
  A brief ending few sentences to wrap up the subsection
\end{itemize}

\hypertarget{data-wrangling}{%
\subsection{Data wrangling}\label{data-wrangling}}

At present I plan to put the data wrangling in the Shiny app section,
but I might make this its own section.

\hypertarget{elevation-charts-1}{%
\subsection{Elevation charts}\label{elevation-charts-1}}

\hypertarget{length-of-trails}{%
\subsection{Length of trails}\label{length-of-trails}}

\hypertarget{trail-categorization-scheme}{%
\subsection{Trail categorization
scheme}\label{trail-categorization-scheme}}

\hypertarget{points-of-interest-layer}{%
\subsection{Points of interest layer}\label{points-of-interest-layer}}

\hypertarget{waypoints-layer}{%
\subsection{Waypoints layer}\label{waypoints-layer}}

\hypertarget{trail-type-layer}{%
\subsection{Trail type layer}\label{trail-type-layer}}

\hypertarget{engaging-with-the-shiny-app}{%
\subsection{Engaging with the Shiny
app}\label{engaging-with-the-shiny-app}}

Here I'll put instructions for how to use the Shiny app (if necessary).
I will also highlight particularly clever features of the app (if
applicable).

\hypertarget{conclusion}{%
\section{Conclusion}\label{conclusion}}

I'll wrap up the report here, and will cover:

\begin{itemize}
\tightlist
\item
  Recall: motivation
\item
  Review of what I did
\item
  Why this is novel
\item
  Why I'm proud of it
\end{itemize}

\newpage

\bibliographystyle{agsm}
\bibliography{bibliography.bib}

\end{document}
